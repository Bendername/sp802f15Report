In this chapter, we describe the platform we indend to target, the underlying
technology we use to support the development of our game, and finally an
architectural overview of the different components needed to be implemented for
our game. Architecture descriptions are accompanied with diagrams depicted in a
\textit{layered architecture} style. Accompanying text describe diagrams in
more detail where relevant.

\section{Platform} \tododaniel{refactor this}
As the development is targeting a mobile platform, it is sensible to target the one with the largest market share.
According to the International Data Corporation (IDC) in 2014 this platform was Android with a 84.7\% market share\cite{marketshare}.
Android OS has the majority of market share and it is free to develop
for, contrary to Apple's iOS which requires an Apple developer account that
costs 99\$ a year\cite{appledevprogram}.
This leads us to choose the Android platform, specifically Android mobile
phones as the target platform.

\todobrian{formulate differently and ref list of reqs}
While we strive to conform to the hardware specifications of lower-end Android
phones, we only have access to mid- and high-end Android hardware, and as such
can only benchmark performance on such devices.

\section{Game Engines}
For this project we decided against developing our own game engine as it would require too much time to create a viable one, after which we likely would not have had enough time left to create the game. 
Since the scope of this project is focused on creating a game for a mobile platform rather than developing a game engine, we decided to use an existing game engine.

When looking at game engines, it became apparent that a wide variety exists, and looking through all of them would be beyond the scope of this project.
Instead we decided to look at three popular ones: Cocos2D, Unity3D, and Unreal Engine.

According to the survey ``Developer Economics Q3 2014: State of the Developer Nation''\cite{visionmobile-survey}, conducted by Vision Mobile\cite{visionmobile}, Unity3D was used by 47\% of game developers in Q3 of 2014, showing that Unity3D is a very popular game engine. 
In the same survey, Cocos2D and Unreal Engine were used by 19\% and 13\% respectively.

At the time when we were deciding which engine to choose Unreal Engine 4 cost 19\$ a month\cite{unrealFree}, while Unity3D 4 and Cocos2D were both free. 
An attempt was made to try to acquire academic licenses to Unreal Engine so that the group did not need to pay for seven licenses. 
This request was not met in due time, so we decided to discard the Unreal Engine.

This leaves us with the choice between Unity3D and Cocos2D, that both allow development of 2D games for mobile platforms and support for multiple platforms targets.
While the group had no experience using Cocos2D, every member of the group had shipped at least one title using Unity3D, so rather than spending time learning how to use a new tool, we decided to use Unity3D to get off to a quicker start.
Unity3D supports programming using the C\# language which most group members felt confident using, hence this would also contribute to a faster development of the game.

% rød tråd til subsection Unity 
\subsection{Unity3D}
Unity3D is a general purpose game engine designed to make it easy to make games
of many different genres. Unity3D consists of an built-in editor, making it
easy to design and change game environments. Using the editor, the user can
place game objects, and attach components to them, such as colliders and
scripts. The editor affords an intuitive approach to designing game-worlds, as
the game can be viewed in 3D or 2D during design and in real-time while the
game is running.
\todobrian{insert img of editor}

In some cases, the editor might not afford enough flexibility, in which case
the Unity3D scripting API is available alongside the editor.
Scripts can be written in three different programming languages:

\begin{itemize}
    \item \textbf{Boo:} a language developed by Unity Technologies, the
        developers of Unity3D. Boo resembles Python.
    \item \textbf{UnityScript:} a language also developed by Unity
        Technologies, resembling JavaScript.
    \item \textbf{C#:} using the Mono framework.
\end{itemize}

All three languages can be used interchangeably within the same project.
\\
\\
\todobrian{find refs}
Unity3D provides frameworks and components for various game-development related
tasks:
\begin{itemize}
    \item 2D and 3D physics simulation.
    \item 2D and 3D graphics rendering.
    \item Networking.
    \item Asset management and loading.
\end{itemize}

\todobrian{find refs}
Physics simulations are using Nvidia's PhysX physics engine, and rendering is
offered by either Direct3D or OpenGL/ES.
% What is unity
% What can it do for us
% architecture diagram

\section{Game Architecture Overview}
% Diagram
% Beskriv diagram
% Hvorfor ikke class diagram 
% Details omkring diagram components er i module sections
