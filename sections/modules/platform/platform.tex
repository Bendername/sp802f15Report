As the development is targeted at a mobile platform, it makes sense to develop for the largest of these. 
According to the International Data Corporation (IDC) in 2014 this platform was Android with a 84.7\% market share\cite{marketshare}.
Since the Android OS has the majority of market share and it is free to develop for, contrary to Apple's iOS which requires an Apple developer account that costs 99\$ a year\cite{appledevprogram}, it was decided to target this platform.


\section{Game Engines}
For this project we decided against developing our own game engine as it would require too much time to create a viable one, after which we likely would not have had enough time left to create the game. 
Seeing as the scope of this project is more focused on creating a game for a mobile platform rather than developing a game engine, we decided to use an existing game engine.

When looking at game engines there is a wide variety of these available to choose from, and looking through all of them would take up too much time. 
Instead we decided to look at a few popular ones: Cocos2D, Unity3D, and Unreal Engine.

According to the survey ``Developer Economics Q3 2014: State of the Developer Nation''\cite{visionmobile-survey}, conducted by Vision Mobile\cite{visionmobile}, Unity3D was used by 47\% of game developers in Q3 of 2014, showing that Unity3D is a very popular game engine. 
In the same survey, Cocos2D and Unreal Engine were used by 19\% and 13\% respectively.

At the time when we were deciding which engine to choose Unreal Engine 4 cost 19\$ a month, while Unity3D 4 and Cocos2D were both free. 
An attempt was made to try to acquire academic licenses to Unreal Engine so that the group did not need to pay for seven licenses. 
This would not be possible immediately so instead of waiting we decided to discard the Unreal Engine.

The choice came down to Unity3D and Cocos2D, which both allow development of 2D games for mobile platforms and they both support multiple platforms. 
While the group had no experience using Cocos2D, every member of the group had shipped at least one title using Unity3D, so rather than spending time learning how to use a new tool, we decided to opt for the known tool to get off to a quicker start. 
As well, Unity3D supports programming using the C\# language which most group members felt confident using, hence this would also contribute to a faster development of the game.