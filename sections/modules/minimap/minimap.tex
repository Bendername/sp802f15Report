\todobrian{I know this should not go here. We need to come up with a structure
for individual implementation modules, like for the levels, minimap, loading
in shit from json etc.}
In this chapter, we explore two different implementation of our in-game
``minimap'' representation. The underlying implementations are similar, however
the initial implementation was developed during prototyping sessions to show
the feasibility and usefulness of having a minimap in-game. This initial
implementation was however encumbered by sub-par performance on mobile devices,
for reasons which we will explore in more depth later in this chapter. The
implementation following was an attempt in mitigating the performance issues,
which proved successful.

\section{Initial implementation}
The minimap implementation was initially inspired by \textit{bitmap-blitting}
techniques used in \textit{older} games, which means that the general idea for
the implementation is to \textit{blit} - or plot - individual pixels to a
texture, corresponding to specific types of tiles or entities. This means that
there are several variable factors we need to consider:

\begin{itemize}
    \item The current position of the player - we need to know which area of
        the map we need to present to the player.
    \item The position of entities within the game - enemies, other players
        etc.
    \item And what pixel-color to use when plotting the background and entities.
\end{itemize}

Having all enemies and players managed in other singleton-accessible instances,
makes it easy to satisfy the two first items on the list. The last item
however, we want to be able to define in a flexible way. This
has to do with the fact that any level can be loaded with any given
texturepack, meaning that walls, ground, grass and so on, appear differently
in the game, depending on the texture pack used - we want to reflect that
difference in the minimap as well. We suggest the following
approaches to this problem:

\begin{itemize}
    \item Iterate over tiles in the texturepack for the current level and find
        the average color for each corresponding tile, or
    \item for each texturepack, define the colors in an external file.
\end{itemize}

While the first approach would have provided us with a more generic and
automated solution, we inadvertently chose the latter as it was quickly
prototyped.
\\
\\
The colors used for plotting entities and the background on the minimap are
defined in a \texttt{PNG} image, accompanying the selected texturepack for that
level. The colors are parsed by index, e.g. the first pixel defines ground
color, second pixel defines grass color and etc.
\\
\\
The entities and background is blitted onto a texture at a shifted position, in
order to represent the players current position on the map. If entities such as
enemies and other players are within the players vicinity and the bounds of the
texture, they are blitted onto the minimap textures as well.
\\
\\
As mentioned in the beginning of this chapter, the implementation we just
described proved to have performance implications, which we
partly expected. This has to do with the performance overhead of blitting - or
changing pixel information - on a textures. Whenever a texture changes and
those changes has to be reflected in the screen rendering, the instances method
\texttt{Apply()} has to be called on the texture. Calling \texttt{Apply()}
\textit{uploads} the texture to the GPU (graphics processing unit) - a
potentially expensive operation according to Unity3D's documentation, and shown
by our own profiling and testing.

\section{Optimized implementation}
In order to optimize our minimap it was important to avoid the recalculation of texture each time something changed.
To do so a new approach was applied. 
The new approach is based on offsetting the texture instead of recalculate a new texture all the time.
By doing so a lot time is saved on the CPU and transferred to the GPU instead. 
This is done by adding a \textit{Raw Image} to a \textit{game object} with a canvas renderer.
The \textit{Raw Image} contains the properties:
\begin{itemize}
\item Texture
\item Color
\item Material
\item UV Rectangle
\end{itemize}
Where the texture will be our full map, and the UV Rectangle will be used to offset and control the size of will be shown in the minimap.
\\\\
The task which the CPU is still required to do is:
\begin{itemize}
\item Calculate the offset that is required for the player to be inside the minimap view screen
\item Plotting entities on the texture
\end{itemize}
To calculate the offset to keep the player in minimap view screen, the position of the players character is used.

The UV Rectangle, which controls this process, is controllable by changing the values:
\begin{itemize}
\item Height (H) - Height of the what should be shown, represented in a ratio of
, a width (W), a X-offset(X) and Y-offset (Y).
\end{itemize}

Since the offset The offset is simply calculated from taking the player position relative to the width and height 
\todoanders{Anders make this better and finish it.}