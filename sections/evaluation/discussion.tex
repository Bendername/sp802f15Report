In this section we will discuss the user experience of our game, the performance issues encountered, as well our extensible system.

\subsection{User Experience}
As shown in the user test of the game(section \ref{test:results}) we had mostly positive feedback in terms of the gameplay and features of the game.
Most users took the game as an early beta, and accepted the limited content and the fact that they were testing the early stages of a game.

One thing to note however, is that test groups were instructed in the contents of the game before actually playing it.
This is of course because the focus has been on the actual game mechanics, and not so much the user interface part.

During tests it became apparent that some users who were not aware that objectives actually existed in the game.
This is obviously a problem because it is a concept, which serves to make the game cater to hardcore players, and if we cannot communicate this, the game will not be as interesting as it could have been for our target group.

\subsection{Performance}

\subsection{Extensibility}
Creating a game from a programming perspective is not only about manufacturing the game itself, but also about making the systems that facilitate the creation of the game.
The game developed is more so a tech demo than a fully fledged game in terms of added content.
In most cases it is trivial to extend the content of the game. 
As such the game is not at all complete in terms of content, but it is very strong in terms of the framework and architecture in place to cater for the goals we set out to meet, i.e. making a game for hardcore players.

\subsubsection{Traits}\label{dicsussion:traits}
The trait system is used to create traits that players receive upon leveling up.
Due to the flexibility of our trait system, see section \ref{sec:modules:traits}, it is trivial to add more traits which manipulate any given attribute in the game, be it damage, movement speed or adding an entirely new effect. 

Character classes are also differentiated by their individual traits.
Mendy is a healer, because she has a \emph{healing} trait.
These are also defined in the trait system.
As such it is also trivial to add more classes or differentiate the ones already added if desirable.

\subsubsection{Missions}\label{dicsussion:missions}
For creating missions, maps and waves the mission system is used.
The system relies on .png files used to generate the map, and .json files to define the missions and waves, as well as which enemies and objectives encountered in these.
Using these ubiquitous formats, we have been able to open up these systems, so that users may create custom content.
The .json files, while human readable, tends to become large and thus hard to comprehend.
In addition to this, its precise functionality and syntax is vague without knowing the internal representation.
As a result, the user may find it difficult to create valid content that can be loaded into the game.
In contrast the way maps are created is by drawing an image of it and using colors to determine which type of tiles should be placed where. This way of designing it is easier to read as the map generated from it will look the same only with different colors.

In addition, the missions have support for \textit{MissionObjectives}, see figure \ref{fig:json_chart}, which are currently not implemented into the game.

\subsection{Workflow}

\subsection{Unity3D}
Unity3D enabled us to develop a game from early design to prototyping to a
finished game. We would like to mention that we find Unity3D to be an
impressive technology that enables quick development and early results. By not
having \textit{middleware} such as Unit3D can require significantly more work
before a functional game comes from it.
Having said that, Unity3D did present to us inconveniences alongside the
advantages. In the following, we describe some of the \textit{pros} and
\textit{cons} we experienced with working with Unity3D.

\subsubsection*{Positives}
%pros:
%profiler in free-edition, easy-to-use,
%rpc > low-level socket programming
Bundled with Unity3D is feature-rich profiler, capable of profiling GPU, CPU
and memory usage among other things - a valuable tool when attempting to
identify performance bottlenecks. Up until recently, the profiler has only been
available in the pro edition of Unity3D. With the release of Unity3D 5.0, the
profiler is bundled with all editions of Unity3D. Unity3D 5.0 was released
during this project, which made the profiler available to us. This allowed us
to perform extensive performance analysis of different components within our
architecture, such as AI, networking. This has in more than one occasion
enabled us to find significant performance bottlenecks.
\\
\\
Unity3D provides an API for RPC programming, removing the need for low-level
socket programming. Much like Unity3D's other components offer quick
development, the RPC API made it possible for us to quickly prototype the
networked aspect of our game - an aspect with high uncertainties at the
beginning of the project due to limited network programming experience
among the members of the group.

\subsubsection*{Negatives}
%cons:
%only one socket available in non-unity. Inverted  rects y axis, inconsistencies in documentation.
%noget shit om rpc
\todobrian{write also about socket issue and inverted y-rekt?}

\todobrian{insert refs}
Even though Unity3D has been in development for approximately 9 years, it is
still surprisingly unstable in certain situations. It is not uncommon that the
Unity3D editor process has to be \textit{force killed}. This is especially a
reoccurring problem on OSX and the exact cause for this is unknown to us. This
had a negative effect on development, as it would often be required that Unity3D
be restarted.
\\
\\
\todobrian{Is this correct?? Or was it only with 3 input source - whatever that
means?}
Unity3D supports various input devices, but we did experience problems with
changing or adding input devices to the game during runtime, e.g. attaching a
gamepad to a PC while the game is running. There were also inconsistencies in
axis- and button-mapping among different operating systems. To circumvent this,
we used a 3rd-party library. In the meantime, Unity Technologies have updated
Unity3D to, presumably, offer the same capabilities.
\\
\\
It is not possible to construct \textit{Nested} prefabs within Unity3D. Prefabs
within Unity3D is the concept of creating a \textit{template} of a game object,
pre-configured with components. So to state the problem differently, it is not
possible to nest such templates within each other. Evidently, this did not
impede development, but it did present inconveniences that required
workarounds.
\todobrian{write something about swaapping dlls - I dunno this problem}



%\subsection{DO NOT READ BELOW THIS}

%
%This has also been the case with our game.
%The current framework supports an advanced set of mechanics; the trait system is the basis of all character customization, be it player based or enemy based.
%As such, it is easy to develop new types of player classes or enemies with certain abilities, it is merely a question of creating more effects, wrapping them up in a trait, and adding them to a player class or enemy.
%Our mission system is also extensible.
%It is possible to create custom missions with a set of waves, enemies and objectives.
%The system is made such that it is easy to make new and interesting missions which can cater to different types of players. 

%Another place for improvement is performance.
%It is a tall order to have a large environment with hordes of intelligent enemies, and on top of that have it run smoothly on mobile devices.
%In order to have the game contain more stress points, it is essential that the game supports many types of enemies with many types of attack strategies.
%In order to do this, one could have more agent states in the enemies, making sure that resources are only spent when they are directly interacting with a player.
%This has already been done to some extent, but could be optimized even further.


%Exntensibility:
%	Trait system
%	Mission System
%	GUI/tools for custom content

%Performance
%	AI/Enemies
%	Networking
%		server does everything \& mobile being server

%UX Daniel - 
%	GUI is shoit
%		How to control shit? reference tests

