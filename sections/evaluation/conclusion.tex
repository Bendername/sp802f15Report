We began with the problem statement:
\begin{center}
\textit{What is required in order to create a game, with gameplay which caters to hardcore gamers on a mobile device?}
\end{center}

The major obstruction in this, is satisfying all the conditions mentioned in Table \ref{tab:relevantFactors}.

In order to do this we included concepts of character classes (section \ref{gamedesign:ourgame:classes}), traits (section \ref{gamedesign:ourgame:traits}), different enemies(section \ref{gamedesign:ourgame:enemies}), and a crafting system (section \ref{gamedesign:ourgame:weapons}) with a corresponding resource system (section \ref{gamedesign:ourgame:crafting})
Furthermore, we included a competitive element both internally between players in terms of individual performance, and externally between player groups in terms of achieving the highest score and in terms of actually beating the game(section \ref{gamedesign:ourgame:scoring}).

During testing, some testers mentioned that our game had some balancing issues. 
This inherently collides with the goal of keeping \emph{players in the zone} as described in section \ref{gamedesign:science}. 

Despite this, the post-play questionnaire in section \ref{test:results} clearly shows a mostly positive attitude towards the game by our target group.
From this we gather that the game successfully solves the task we set out to do in terms of catering to the hardcore players.\\

The other success criteria for our game has to do with performance, as explained in Section \ref{successcriteria}.
We set out to make the game run on mobile devices, and as such it should be optimized for the hardware which is available on that platform. 
Performance was a priority throughout development, and most problems within this area mostly sorrounded either improving the framerate, or decreasing the network bandwidth usage.

As shown in Section \ref{test:performancetest} the game runs smoothly and well above our targeted FPS.
This was important because it helps keeping the game \emph{fun} and \emph{engaging} because there are no disruptions, as elaborated in section \ref{gamedesign:science}.
As such the performance goals has been fulfilled almost completely.