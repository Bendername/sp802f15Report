We set out to create a game for mobile devices which caters to the more hardcore segment of players. The major obstruction in this, is making a complex game which at the same time runs on a mobile device. Generally making  game complex means creating multiple layers of gameplay in the game, in our case this included character classes (section \ref{gamedesign:ourgame:classes}), traits(section \ref{gamedesign:ourgame:traits}), different enemies(section \ref{gamedesign:ourgame:enemies}), complex crafting system(section \ref{gamedesign:ourgame:weapons}) with a corrosponding resource system(section \ref{gamedesign:ourgame:crafting}) and finally a competative element both internally between players, in terms of individual performance, and externally between player groups in terms of the highest achieved score, and in terms of actually beating the game(section \ref{gamedesign:ourgame:scoring}).\tododennis{Make sure this is reflected in this section}

The game developed is more so a tech demo than a fully fledged game in terms of added content. In most cases it is trivial to extend the content of the game. Due to the flexibility of our trait system\tododennis{Kasper: ref til din trait section}, it is trivial to add more \emph{traits} to the game, which manipulate any given attribute in the game, be it damage, movementspeed or adding an entirely new effect. 

Character classes are also differentiated by their individual abilities. Mendy is a healer, because she has a \emph{healing} ability. These are also defined in the trait system. As such it is also trivial to add more classes or differentiate the ones already added, if that is desireable.

As such the game is not at all complete in terms of content, but it is very strong in terms of the framework and architecture in place to cater for the goals we set out to meet, i.e. making a game for hardcore players. \\

As shown in the user test of the game(section \ref{test:results}) we had mostly positive feedback in terms of the gameplay and features of the game. Most users took the game as an early play, and accepted the limited content and the fact that they were testing the ''framework'' upon which the game would be build.

One thing to note, however, is that the test group was instructed in the contents of the game before actually playing it. This is ofcourse because the focus has been on the actual technical part of the game, and not so much the user interface part. This also came to show by some users who were not aware that objectives actually existed in the game. This is obviously a problem, because it is a concept which serves to make the game cater to hardcore players, and if we cannot communicate this the game will not be as interesting as it could have been for our target group.

According to some testers, our game also had some balancing issues. This inheritly collides with the goal of keeping \emph{players in the zone} as described in section \ref{gamedesign:science}. Although this is important, it is also something which is normally dealt with in the latter parts of developing a game.

Overall though the game successfully solves the task we set out to do in terms of catering to the hardcore players. The post-play questionaire in section \ref{test:results} clearly shows a mostly positive attitude towards the game by our target group.\\

The other success criteria for our game, has to do with performance. We set out to make the game run on mobile devices, and as such it should be optimized for the hardware which is available on that platform. A lot of problems arose during development, but performance was a priority throughout development. This problem was mostly twofold:

\begin{itemize}
\item Make sure the game runs smoothly in terms of FPS 
\item Make sure the game run smoothly in a networked environment
\end{itemize}

As shown in \tododennis{ref to the hardware performance tests} the game runs smoothly at X\tododennis{insert fps} fps. This was important because it helps keeping the game \emph{fun} and \emph{engaging} because there is no disruptions, as elaborated in section \ref{gamedesign:science}.  So the performance goals has been fulfilled almost completely. 

