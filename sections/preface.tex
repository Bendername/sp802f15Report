%\\
%\\
%\vspace{\baselineskip}\hfill Aalborg University, \today
%\vfill\noindent
\cleardoublepage%
\begin{minipage}[b]{0.45\textwidth}
 \centering
 \rule{\textwidth}{0.5pt}\\
  Brian Frost Pedersen\\
 {\footnotesize <bpeder10@student.aau.dk>}
 \centering
 \rule{\textwidth}{0.5pt}\\
  Dennis Bæk Nielsen\\
 {\footnotesize <dbni11@student.aau.dk>}
 \centering
 \rule{\textwidth}{0.5pt}\\
  Daniel Steinar Fridjonsson\\
 {\footnotesize <dfridj11@student.aau.dk>}
 \centering
 \rule{\textwidth}{0.5pt}\\
  Michael Lausdahl Fuglsang\\
 {\footnotesize <mfugls11@student.aau.dk>}
\end{minipage}
\hfill
\begin{minipage}[b]{0.45\textwidth}
 \centering
 \rule{\textwidth}{0.5pt}\\
  Anders Bender\\
 {\footnotesize <abende11@student.aau.dk>}
 \centering
 \rule{\textwidth}{0.5pt}\\
  Benjamin Hubert\\
 {\footnotesize <bhuber11@student.aau.dk>}
 \centering
 \rule{\textwidth}{0.5pt}\\
  Kasper Lind Sørensen\\
 {\footnotesize <klsa11@student.aau.dk>}
\end{minipage}
\vspace{3\baselineskip}

\cleardoublepage
\chapter*{Preface\markboth{Preface}{Preface}}\label{ch:preface}
\addcontentsline{toc}{chapter}{Preface}
As our report differ from traditional reports in terms of structure, following will be a reading guide.
In addition, we here define the terminology we use throughout the report.

\section*{Readers Guide}\label{preface:readersguide}
The report is split into three parts.
Part I features an introduction and the design.
It serves to define a problem statement, describe our workflow, and the design our game.

Part II describes the implementation of our game.
It is split into chapters that describe major modules of our game in regards to obstacles encountered during the development, and how these modules are implemented.
However, it should be emphasized that the order of these chapters does not represent the order which each system was developed.
This is because some of the systems are developed in parallel and other are being developed and then reworked later in the process.

Part III reflects on the game and the process of developing it. It describes the tests that were conducted during development as well as the results. A discussion of the choices made during the development is brought up and a conclusion on the project presented.

\section*{Terminology}\label{preface:terminology}
\textbf{Gamepad}\vspace{4pt}\\
A game input device commonly used with consoles.
Examples include the X-Box controller and the Playstation controller.\\
\\
\textbf{Mobile Device}\vspace{4pt}\\
A portable computing device such as a smartphone or tablet computer.\cite{mobileOx}\\
\\
\textbf{Gameplay}\vspace{4pt}\\
Throughout this report, we use Sid Meier's definition; \textit{``A series of interesting choices''}\cite{GDC2012}.
The choice of this definition is argued in section \ref{sec:specifyingtheproblemstatemen}.\\
\\
\textbf{Strategy}\vspace{4pt}\\
We define the strategic aspect of a video game as \emph{a series of meaningful choices which the player/players believe will take the person/persons closer to the goal.}
The choice of this definition is argued in section \ref{sec:specifyingtheproblemstatemen}.\\
\\
\textbf{Balancing}\vspace{4pt}\\
Altering the difficulty of the game to neither be too hard, nor too easy.
The idea is to keep the player in the \textit{Flow zone}, see section \ref{makingthegamefun:cognitiveflow}.\\
\\
\textbf{Shooter}\vspace{4pt}\\
A genre of video games where the main objective is to shoot foes or targets. \cite{oxShooter}\\
\\
\textbf{FPS}\vspace{4pt}\\
Acronym for \textbf{F}rames \textbf{P}er \textbf{S}second.
The amount of frames rendered per second.\\
\\
\textbf{Frame drop}\vspace{4pt}\\
An occasional and consistent decline in FPS. Although the term is not plural, it could be the case that several sequential frames are not rendered.
To the observer, this would be perceived as the application or animations ``freezing''.\\
\\
\textbf{Middleware}\vspace{4pt}\\
A software layer offering convenient access to a software layer below it, often with an increased level of abstraction.
An example is Unity3D offering rendering capabilities through OpenGL/Direct3D without having the programmer understand the API for such rendering backends.\\
\\
\textbf{MMORPG}\vspace{4pt}\\
Acronym for \textbf{M}assively \textbf{M}ultiplayer \textbf{O}nline \textbf{R}ole-\textbf{P}laying \textbf{G}ame.
Examples include World of Warcraft and Everquest.\\
\\
\textbf{GameEngine}\vspace{4pt}\\
A Game Engine is a framework for common tasks associated with games programming.
An example of a Game Engine is Unity3D, which simplifies tasks such as rendering 3d graphics and networking.
Furthermore it has an editor that further simplifies the task of creating game worlds\cite{unityGameEngine}.\\
\\
\textbf{Gameobject}\vspace{4pt}\\
Base class for all entities in Unity.\cite{prefaceGameobject}\\
\\
\textbf{Prefab}\vspace{4pt}\\
A gameobject configured and attached several components. The prefab can be used as a ``template'' to instantiate new gameobjects with the same configuration of components.
An analogy to OOP terminology would be a class.\\
\\
\textbf{Collider}\vspace{4pt}\\
A Collider in Unity is a component that can be attached to a Gameobject, such that a method is called when another Gameobject with a Rigidbody\cite{rigidbody} collides with it.\cite{collider2d}
A variation of this is called Trigger Collider.
A Trigger Collider does not collide with Rigidbodies, but instead responds when a Rigidbody is inside the volume of the trigger collider.\cite{collider2dtrigger}\\
\\
\textbf{Raycast}\vspace{4pt}\\
Casts a ray against colliders in the game. \cite{raycast}\\
\\
\\
\textbf{RPC}\vspace{4pt}\\
Remote Procedure Call - A networked function call on a remote machine. \cite{rpc}\\
\\