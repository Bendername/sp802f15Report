As the development is targeted at a mobile platform, it makes sense to develop for the largest of these. 
According to the International Data Corporation (IDC) in 2014 this platform was Android with a 84.7\% market share\cite{marketshare}.
Since the Android OS has the majority of market share choosing this as the target platform is an easy choice, but including others is ideal.
Therefore we look to existing game engines rather than a library, in order to ease cross platform development.

\section{Game Engines}
Choosing the right engine is important, as it will not only be a limiting factor in what is possible within the game, but choosing a good engine will severely ease development.
The criteria for the game engine is therefore as follows \todomichael{Are we limiting ourselves to 2d graphics?}

\begin{center}
	\begin{tabular}{| c | c | c |}
		\hline
			 & Unity & Unreal Engine \\ \hline
		Networked realtime multiplayer & \checkmark & \checkmark \\ \hline
		2D graphics & \checkmark & \checkmark \\ \hline
		Gamepad support & \checkmark & \checkmark \\ \hline
		cross platform compilation & \checkmark & \checkmark \\ \hline
	\end{tabular}
\end{center}

As can be seen, both these commercial engines fulfill the criteria, so the choice falls down to our preference.
Since the group has worked with the Unity engine in a previous semester, and as we are more familiar with C\#, the choice of engine will be Unity.
