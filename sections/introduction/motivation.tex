\section{Motivation} \label{sec:motivation} \tododaniel{Motivation er like 25\% motivation 75\% Gamer Dedication Score}
Presently the most played category of games for mobile devices is casual games.
According to ESA\cite{ESA}\cite{ESApdf} (\textit{Entertainment Software Association}), 46\% of the mobile gamers play casual games, not including the 31\% that play a combination of puzzle-, board-, game show-, trivia- and card-games.

But what is a casual game exactly?
Casual games are games that cater to casual gamers.\tododaniel{obvious sentence, kan vi ikke skrive noget andet her.}
In an article from Gamasutra, a proposed way to measure how \textit{casual} or \textit{hardcore} a gamer is, is through a questionaire concerning 15 factors\cite{casual_vs_hardcore}.
These 15 factors and their weight according to Gamasutra can be seen in Table \ref{tab:gamerdedication}\cite{casual_vs_hardcore}.
For each factor, a game can answer a number from 1 to 5, where 1 means he thinks the factor does not fit him, and a 5 meaning he thinks the factor fits him very well.
Through these numbers, a \emph{Gamer Dedication Score} or \emph{GD} can be calculated using Equation \ref{eq:GD} where $w$ is the weight, $s$ is the score and $n$ is 15 (because there is 15 factors).

\begin{equation}\label{eq:GD}
GD = \frac{\sum\limits_{j=1}^n w_j s_j}{\sum\limits_{j=1}^n 5 w_j}
\end{equation}

The result is a percentage that through Table \ref{tab:gamertype} can be used to determine the type of a gamer a person is\cite{casual_vs_hardcore}.
Not all of these factors can be directly influenced when designing a game, and not all of them are relevant when designing our game.

Below we list the factors mentioned in the article from Gamasutra, as well as shortly mentioning whether we find a particular factor relevant in our case, when designing our game:

\begin{enumerate}
\item \emph{Playing games over many long sessions.}
The game should encourage the gamer to play longer through reward or progress, therefore it is relevant.

\item \emph{Discussing games with friends/bulletin boards.}
The game should encourage games to discuss the game outside the boundaries of the game itself, such as on forums.
We have no intention of creating a bulletin board or chat-system within our game, and therefore we deem this an irrelevant factor.

\item \emph{Having comparative knowledge of the industry.}
Hardcore games tend to show greater knowledge of the gaming industry than more casual games.
Knowledge of the industry happens outside of the game and it not easily affected by a game. We deem it irrelevant in our case.

\item \emph{Being tolerant of frustration.}
Hardcore games tend to be more tolerant to games of higher difficulty.
While \textit{higher difficulty} is subjective, one can affect this factor by adjusting the difficulty of the game.
We find it relevant.

\item \emph{Showing early adoption behaviour.}
While hardcore gamers tend to be \textit{early adoptors}, it is not a factor that can be influenced by a game itself.
We find it irrelevant in our case.

\item \emph{Desiring to modify or extend games in a creative way.}
The game engine can be made with modification or extension in mind, allowing for players to create their own content or features, therefore it is relevant.

\item \emph{Having technological savvy.}
Technological savvy is used to access the game, once it is accessed it is up to the game to be user friendly, therefore it is irrelevant.

\item \emph{Having the latest high-end computers/consoles.}
Since we are developing for mobile platforms, and they generally have low-end hardware, this is irrelevant.

\item \emph{Playing for the exhilaration of defeating (or completing) the game.}
The game can offer the player a reward for completing the game, therefore it is relevant.

\item \emph{Hungering for gaming-related information.}
Gathering information about the gaming world happens outside of the game, therefore it is irrelevant.

\item \emph{Being engaged in competition with oneself, the game and other players.}
The game can provide the player with the ability to challenge other players through various means. Therefore this is relevant.

\item \emph{Being willing to pay for the game.}
Since the game we are developing is a university project it will be free, so this is not relevant.

\item \emph{Preferring games that have depth and complexity.}
The game can directly cater to this preferrence by having deep and complex gameplay, therefore this is relevant.

\item \emph{Time started playing games relative to the age of the industry.}
The game cannot affect this, therefore it is not relevant.

\item \emph{Preferring violent/action games.}
Theming the game as a violent action game will cater to this preferences, therefore it is relevant.
\end{enumerate}

The relevant factors are then taken into consideration when designing our game.

\begin{table}[H]
\begin{tabular}{|l|l|l|}
\hline
\rowcolor[HTML]{C0C0C0} 
Factor                                                                & Weight 	& Relevancy \\ \hline
Playing games over many long sessions                                 & 10     	& Relevant 	\\ \hline
Discussing games with friends/bulletin boards                         & 10     	& Irrelevant\\ \hline
Having comparative knowledge of the industry                          & 10      & Irrelevant\\ \hline
Being tolerant of frustration                                         & 9       & Relevant 	\\ \hline
Showing early adoption behaviour                                      & 9       & Irrelevant\\ \hline
Desiring to modify or extend games in a creative way                  & 8       & Relevant  \\ \hline
Having technological savvy                                            & 7       & Irrelevant\\ \hline
Having the latest high-end computers/consoles                         & 7       & Irrelevant\\ \hline
Playing for the exhilaration of defeating (or completing) the game    & 7       & Relevant  \\ \hline
Hungering for gaming-related information                              & 6       & Irrelevant\\ \hline
Being engaged in competition with oneself, the game and other players & 6       & Relevant  \\ \hline
Being willing to pay for the game                                     & 5       & Irrelevant\\ \hline
Preferring games that have depth and complexity                       & 3       & Relevant  \\ \hline
Time started playing games relative to the age of the industry        & 2       & Irrelevant\\ \hline
Preferring violent/action games                                       & 1       & Relevant  \\ \hline
\end{tabular}
\caption{Table of factors describing the gamer dedication, their weight and whether they are relevant from a game design perspective.}
\label{tab:gamerdedication}
\end{table}

\begin{table}[H]
\begin{tabular}{|l|c|}
\hline
\rowcolor[HTML]{C0C0C0} 
Gamer type             & $GD$     \\ \hline
Ultra Casual/non-gamer & 0 - 29   \\ \hline
Casual                 & 30 - 45  \\ \hline
Transitional/moderate  & 46 - 55  \\ \hline
Hardcore               & 56 - 70  \\ \hline
Ultra hardcore         & 71 - 100 \\ \hline
\end{tabular}
\caption{Type of gamer derived from $GD$.}
\label{tab:gamertype}
\end{table}

%These factors give an indication of how to categorize players, to which we ask the question: 
%\begin{center}\label{intro:problem_statement}
%\textit{What is required in order to create a game, with gameplay which caters to hardcore gamers on a mobile device?}
%\end{center}
