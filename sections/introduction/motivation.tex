\section{Motivation} \label{sec:motivation}
Nowadays the most played category of games for mobile devices is casual games.
According to ESA\cite{ESA}\cite{ESApdf} (\textit{Entertainment Software Association}), 46\% of the mobile gamers play casual games, not including the 31\% that play a combination of puzzle-, board-, game show-, trivia- and card-games.

But what is a casual game exactly?
Casual games are games that cater to casual gamers.
In an article from Gamasutra, a proposed way to measure the \textit{casualness} of a player is through a questionaire concerning 15 factors\cite{casual_vs_hardcore}.
These 15 factors and their weight according to Gamasutra can be seen in Table \ref{tab:gamerdedication}\cite{casual_vs_hardcore}.
For each factor, a player can answer a number from 1 to 5, where 1 means he/she thinks the factor doesn't fit him/her, and a 5 meaning he/she thinks the factor fits him/her very well.
Through these numbers, a \emph{Gamer Dedication Score} or \emph{GD} can be calculated using Equation \ref{eq:GD} where $w$ is the weight, $s$ is the score and $n$ is 15.

\begin{equation}\label{eq:GD}
GD = \frac{\sum\limits_{j=1}^n w_j s_j}{\sum\limits_{j=1}^n 5 w_j}
\end{equation}

The result is a percentage that through Table \ref{tab:gamertype} can be used to determine the type of gamer the player is\cite{casual_vs_hardcore}.
\todobenjamin{Find en grund til at lave hardcore spil.}
Not all of these factors can be directly influenced by the game, and since we are developing a hardcore game the irrelevant factors and be filtered out.

\emph{Playing games over many long sessions.}
The game can encourage the player to play longer through reward or progress, therefore it is relevant.

\emph{Discussing games with friends/bulletin boards.}
Bulletin boards happen outside of the game, therefore it is irrelevant. \todo{Eller hvad? Man kan vel have et forum ingame?}

\emph{Having comparative knowledge of the industry.}
Knowledge of the industry happens outside of the game, therefore it is irrelevant.

\emph{Being tolerant of frustration.}
The game can increase in difficulty to increase frustration levels, therefore it is relevant.

\emph{Showing early adoption behaviour.}
Early adoption behaviour happens before the game is played, therefore it is irrelevant.

\emph{Desiring to modify or extend games in a creative way.}
The game engine can be made with modification or extension in mind, allowing for players to create their own content or features, therefore it is relevant.

\emph{Having technological savvy.}
Technological savvy is used to access the game, once it is accessed it is up to the game to be user friendly, therefore it is irrelevant.

\emph{Having the latest high-end computers/consoles.}
Since we are developing for mobile platforms, and they generally have low-end hardware, this is irrelevant.

\emph{Playing for the exhilaration of defeating (or completing) the game.}
The game can offer the player a reward for completing the game, therefore it is relevant.

\emph{Hungering for gaming-related information.}
Gathering information about the gaming world happens outside of the game, therefore it is irrelevant.

\emph{Being engaged in competition with oneself, the game and other players.}
The game can provide the player with the ability to challenge other players through various means. Therefore this is relevant.

\emph{Being willing to pay for the game.}
Since the game we are developing is a university project it will be free, so this is not relevant.

\emph{Preferring games that have depth and complexity.}
The game can directly cater to this preferrence by having deep and complex gameplay, therefore this is relevant.

\emph{Time started playing games relative to the age of the industry.}
The game cannot affect this, therefore it is not relevant.

\emph{Preferring violent/action games.}
Theming the game as a violent action game will cater to this preferrence, therefore it is relevant.

This produces a shorter list seen in Table \ref{tab:relevantFactors}, which can be used to make game design decissions in order to cater to hardcore gamers.

\begin{table}[h]
\begin{tabular}{|l|l|}
\hline
\rowcolor[HTML]{C0C0C0} 
Factor                                                                & Weight \\ \hline
Playing games over many long sessions                                 & 10     \\ \hline
Discussing games with friends/bulletin boards                         & 10     \\ \hline
Having comparative knowledge of the industry                          & 10     \\ \hline
Being tolerant of frustration                                         & 9      \\ \hline
Showing early adoption behaviour                                      & 9      \\ \hline
Desiring to modify or extend games in a creative way                  & 8      \\ \hline
Having technological savvy                                            & 7      \\ \hline
Having the latest high-end computers/consoles                         & 7      \\ \hline
Playing for the exhilaration of defeating (or completing) the game    & 7      \\ \hline
Hungering for gaming-related information                              & 6      \\ \hline
Being engaged in competition with oneself, the game and other players & 6      \\ \hline
Being willing to pay for the game                                     & 5      \\ \hline
Preferring games that have depth and complexity                       & 3      \\ \hline
Time started playing games relative to the age of the industry        & 2      \\ \hline
Preferring violent/action games                                       & 1      \\ \hline
\end{tabular}
\label{tab:gamerdedication}
\caption{Table of factors describing the gamer dedication and their weight.}
\end{table}

\begin{table}[h]
\begin{tabular}{|l|c|}
\hline
\rowcolor[HTML]{C0C0C0} 
Gamer type             & $GD$     \\ \hline
Ultra Casual/non-gamer & 0 - 29   \\ \hline
Casual                 & 30 - 45  \\ \hline
Transitional/moderate  & 46 - 55  \\ \hline
Hardcore               & 56 - 70  \\ \hline
Ultra hardcore         & 71 - 100 \\ \hline
\end{tabular}
\label{tab:gamertype}
\caption{Type of gamer derived from $GD$.}
\end{table}

\begin{table}[h]
\begin{tabular}{|l|l|}
\hline
\rowcolor[HTML]{C0C0C0}
Factor                                                                & Weight \\ \hline
Play session duration and frequency									  & 10     \\ \hline
Frustration level			                                          & 9      \\ \hline
Modifying the game 									                  & 8      \\ \hline
Exhilaration of defeating (or completing) the game 				      & 7      \\ \hline
Engaging in competition with oneself, the game and other players 	  & 6      \\ \hline
Depth and complexity level					                          & 3      \\ \hline
Violence/action level 		                                          & 1      \\ \hline
\end{tabular}
\label{tab:relevantFactors}
\caption{List of factors the game can influence.}
\end{table}

%Common for casual mobile games is a short game-session time span and often a \textit{progression system} and/or \textit{high score system}. 
%The main focus is to foster these systems and become the best on the leaderboards.
%The multiplayer aspects are asynchronous, where players are not necessarily online at the same time, and mainly sharing your achievements, sharing items or otherwise participating in segregated multiplayer experiences.\\

%Synchronous multiplayer mobile games do exist, such as \textit{GunBros 2} and \textit{Angry Birds Go!}, but such games are often designed around the same casual game concepts as described above. 
%That is, they are to be played in short bursts, utilizing short play-sessions, and also they are to be played many times utilizing the progression system.\\

%The traditional PC/console games are normally games with more advanced controls and expect the player to spend a reasonable amount of time on a game-session. 
This leads to the question:
\begin{center}\label{intro:problem_statement}
\textit{What is required in order to create a non-casual game for mobile devices?}