\section{Success Criteria}\label{successcriteria}
The success criteria for our game is two-fold, user experience and inherently from that; performance.

The target audiences' enjoyment of the game is determined through the user experience.
It follows that whether the game caters to the target audience is decided by the user experience.
In order to test whether our product caters to hardcore players, a group of hardcore players should play the game.
They will then answer a questionnaire that will then convey whether they enjoyed the game.

Performance is important in order to deliver the gameplay in as smooth a fashion as possible.
Frame drops and low FPS are intrusive to gameplay.
This means that the game should have as few frame drops as possible, whilst maintaining the highest FPS as possible.
In order to test this we will run performance tests which focus on determining the average and minimum FPS.
These values should ideally be over a threshold.

The exact threshold is very hard to determine scientifically, because of the factors at play.
What is most important for our game is that there are enough frames for the game to be fluent for the eyes.
The standard computer screen refresh rate is 60hz, and the newer monitors like BenQ's XL series run 120/144hz for high performance gaming\cite{benq}.
As such we define \textit{60 FPS} as the target for PC.

Mobile devices are generally not as powerful as PCs.
This means that their performance can not be expected to be similar.
Therefore, we target a slightly lower FPS for mobile devices.
One could argue 24 FPS is decent, because movies use it.
However, the use of motion blur in movies is more aggressive than it is in video games, hence a slightly higher FPS is desirable\cite{mobilefpsone}\cite{mobilefpstwo}.
We therefore chose a target of \textit{45 FPS} on mobile devices.

\section{Workflow}
First and foremost a game will have to be designed.
Once the design is specified enough to begin implementation, work will begin.
We will be using an agile approach because it will allow us to make rapid changes when encountering problems during development.
Firstly, we will develop the most fundamental systems of our game in parallel.
This is to lay a foundation on which to build our content on, once the fundamental systems are adequately finished and integrated.
The exact priority of these fundamental systems will be explained further in section \todobenjamin{add ref here} REF, after the game design has been determined.	