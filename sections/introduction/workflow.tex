\section{Success Criteria}\label{successcriteria}
The success criteria for our game is two-fold, user experience and inherently from that; performance.

The target audiences' enjoyment of the game is determined through the user experience.
It follows that whether the game caters to the target audience is decided by the user experience.
In order to test whether our product caters to hardcore players, a group of hardcore players should play the game.
They will then answer a questionnaire that will then convey whether they enjoyed the game.

Performance is important in order to deliver the gameplay in as smooth a fashion as possible.
Frame drops and low FPS are intrusive to gameplay.
This means that the game should have as few frame drops as possible, whilst maintaining the highest FPS as possible.
In order to test this we will run performance tests which focus on determining the average and minimum FPS.
These values should ideally be over a threshold.

The exact threshold is very hard to determine scientifically, because of the factors at play.
What is most important for our game is that there are enough frames for the game to be fluent for the eyes.
The standard computer screen refresh rate is 60hz, and the newer monitors like BenQ's XL series run 120/144hz for high performance gaming\cite{benq}.
As such we define \textit{60 FPS} as the target for PC.

Mobile devices are generally not as powerful as PCs.
This means that their performance can not be expected to be similar.
Therefore, we target a slightly lower FPS for mobile devices.
One could argue 24 FPS is decent, because movies use it.
However, the use of motion blur in movies is more aggressive than it is in video games, hence a slightly higher FPS is desirable\cite{mobilefpsone}\cite{mobilefpstwo}.
We therefore chose a target of \textit{45 FPS} on mobile devices.

\section{Development Plan}
To fulfil our problem statement an analysis will be made of how to design an interesting game which would cater to hardcore players.
Based on the analysis a selection of game type will be made, including some general gameplay features.
From this point development of the fundamental systems can begin, these systems will be developed in parallel in smaller groups. 
This will ensure faster development in the beginning of the project but still keep code quality because of pair programming.

When the fundamental systems are done more refined gameplay features will be designed and developed to make the game attractive to hardcore players.

\subsection{Workflow}
An agile approach will be used because it will allow us to make rapid changes when encountering problems during development.

When a feature is suggested, the group will discuss if the feature should be in the game.
If it is decided we want the feature in the game the feature will be split into tasks and added to a to-do list.
Tasks will be available for everyone and if a person choose a task that same person will be responsible documenting the task.

Every Friday the game should be played by the group in order to test the new game features and the game in general.
The new features will then be evaluated and bugs will be added to the to-do list.
