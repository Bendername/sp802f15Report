\section{Success Criteria}\label{successcriteria}
The success criteria for our game is two-fold, user experience and inheritly from that; performance.
The user experience is important because it is what dictates whether the game caters to the target audience.
If the target audience does not enjoy the game, then the game does not cater to that audience.
In order to test whether our product caters to hardcore players, a group of hardcore players should play the game.
They will then answer a questionnaire that will then convey whether they enjoyed the game.

Performance is important in order to deliver the gameplay in as smooth a fashion as possible. That is, the game should attempt to run as high a framerate as possible to avoid stutter lag. Further it should have as few framedrops as possible, because these are intrusive to the gameplay. In order to test this we will run performance tests which focus on determining the average and minimum frames per second. These values should ideally be over a threshold. 

The exact threshold is very hard to determine scientifically, because many factors play a role. What is most important for our game is that there is enough frames for the game to be fluent for the eyes. The standard computer screen refresh rate is 60hz, and the newer monitors like BenQ's XL series run 120/144hz for high performance gaming.\cite{benq}. Since it is the minimum of the two we define 60hz / fps as the target for PC.

Mobiles are not as powerful as PC obviously. As such their performance is not as high. Although there is no exact science for defining a good frame rate on mobiles, most people seem to agree that anything above 30 is great. One could argue 24fps is decent, because movies use it. Movies use motion blur greatly though, which is not as aggressive in gaming, hence a slightly higher framerate is desirable.\cite{mobilefpsone}\cite{mobilefpstwo} 
In order to attempt to keep out game above 30fps at all times we define our target as \emph{45fps}, as this allows for a frame rate drop of 15 fps to still be within our goal.

\section{Workflow}
To fulfil our problem statement an analysis will be made of how to design an interesting game which would cater to hardcore players.
Based on the analysis a selection of game type will be made, including some general gameplay features.
From this point development of the fundamental systems can begin, these systems will be developed in parallel in smaller groups. 
This will ensure faster development in the beginning of the project but still keep code quality because of pair programming.
An agile approach will be used because it will allow us to make rapid changes when encountering problems during development.

When the fundamental systems are done more refined gameplay features will be designed and developed to make the game attractive to casual players.