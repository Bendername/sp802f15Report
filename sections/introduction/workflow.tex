\section{Success Criteria}\label{successcriteria}
The success criteria for our game is two-fold, user experience and inheritly from that; performance.
The user experience is important because it is what dictates whether the game caters to the target audience.
If the target audience does not enjoy the game, then the game does not cater to that audience.
In order to test whether our product caters to hardcore players, a group of hardcore players should play the game.
They will then answer a questionnaire that will then convey whether they enjoyed the game.

Performance is important in order to deliver the gameplay in as smooth a fashion as possible. That is, the game should attempt to run as high a framerate as possible to avoid stutter lag. Further it should have as few framedrops as possible, because these are intrusive to the gameplay. In order to test this we will run performance tests which focus on determining the average and minimum frames per second. These values should ideally be over a threshold. 

The exact threshold is very hard to determine scientifically, because many factors play a role. What is most important for our game is that there is enough frames for the game to be fluent for the eyes. The standard computer screen refresh rate is 60hz, and the newer monitors like BenQ's XL series run 120/144hz for high performance gaming.\cite{benq}. As such we define 60hz / fps as the target for PC.

Mobiles are not as powerfull as PC obviously. As such their performance is not as high. Although there is no exact science for defining a good frame rate on mobiles, most people seem to agree that anything above 30 is great. One could argue 24fps is decent, because movies use it. Movies use motion blur greatly though, which is not as agressive in gaming, hence a slightly higher framerate is desireable.\cite{mobilefpsone}\cite{mobilefpstwo} 
In order to attempt to keep out game above 30fps at all times we define our target as \emph{45fps}

\section{Workflow}
First and foremost a game will have to be designed.
Once the design is specified enough to begin implementation, work will begin.
We will be using an agile approach because it will allow us to make rapid changes when encountering problems during development.
Firstly, we will develop the most fundamental systems of our game in parallel.
This is to lay a foundation on which to build our content on, once the fundamental systems are adequately finished and integrated.
The exact priority of these fundamental systems will be explained further in section \todobenjamin{add ref here} REF, after the game design has been determined.	