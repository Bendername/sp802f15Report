\section{Success Criteria}
Two things are important to look at here, user experience and peformance.

The user experience is important because it is what dictates whether the game caters to the target audience.
If the target audience does not enjoy the game, then the game does not cater to that audience.
In order to test whether our product caters to hardcore players, a group of hardcore players should play the game.
They will then answer a questionnaire that will then convey whether they enjoyed the game.

Performance.\todobenjamin{Potentielle kilder i kommentar}\todomichael{Actually write this}
%http://www.eurogamer.net/articles/digitalfoundry-2014-frame-rate-vs-frame-pacing
%http://www.overclock.net/a/the-truth-about-fps

\section{Workflow}
First and foremost a game will have to be designed.
Once the design is specified enough to begin implementation, work will begin.
We will be using an agile approach because it will allow us to make rapid changes when encountering problems during development.
Firstly, we will develop the most fundamental systems of our game in parallel.
This is to lay a foundation on which to build our content on, once the fundamental systems are adequately finished and integrated.
The exact priority of these fundamental systems will be explained further in section \todobenjamin{add ref here} REF, after the game design has been determined.	