\section{Specifying the problem statement}
\label{sec:specifyingtheproblemstatemen}
Our problem statement states:

\begin{center}
\textit{What is required in order to create a non-casual game for mobile devices?}
\end{center}

In order to answer this question we must determine what a mobile device is, and how to make the gameplay non-casual.
\textit{Mobile devices} is very loosely defined in many different contexts, this report will follow Oxford Dictionary and define a mobile devices as 
\begin{center}
``A portable computing device such as a smartphone or tablet computer''\cite{mobileOx}
\end{center}

The definition of \textit{gameplay} is likewise differently defined in many different contexts. 
An example could be Oxford Dictionary which defines it as \textit{``The features of a computer game, such as its plot and the way it is played, as distinct from the graphics and sound effects.''}\cite{gameplayOx}. This, however, is also rather loosely defined since the features of a computer game is a very broad term. 
Another definition could be Sid Meier's definition; \textit{``A series of interesting choices''}\cite{GDC2012}. 
This definition is still vague but it is elaborated further in his speak at GDC 2012.\cite{GDC2012}
He in general uses this definition as a tool to look at the gameplay. If there is \textit{a series of interesting choices}, are these satisfying for the player according to some criteria for the game? We use Sid Meier's definition.\\\\

When designing a non-casual game the relevant factors to look at are stated in Table \ref{tab:relevantFactors}.
Since the goal is a non-casual game, gameplay should have \textit{depth} and \textit{complexity}.
If one follows Sid Meier's definition, the game must contain an advanced series of interesting choices, in order to achieve advanced gameplay.

These choices must lead somewhere, or to something, in which the player feels a sense of achievement. If this is not he case, the choices cease to feel interesting. In order to achieve this, a strategical aspect must be present.

The Oxford dictionaries define \textit{strategy} as \textit{``A plan of action designed to achieve a long-term or overall aim''}.\cite{strategyOx}
Giving the player the ability to make meaningful choices would allow for the player to create strategies in order to achieve goals.
These choices should not be ''right'' or ''wrong'' choices, but more so strategical choices. The player should be able to approach a task in different ways, in which one choice over another can give the player an advantage according to the task at hand. 
An example of this could be a talent-/skill-tree in a MMORPG, which gives the player a variety of choices in how to achieve their goal whether it is killing a boss in the game or battling other players.
As such, we define the strategic aspect of a video game as \emph{a series of meaningful choices which the player/players believe will take the person/persons closer to the goal.}

\begin{table}[h]
\begin{tabular}{|l|l|}
\hline
\rowcolor[HTML]{C0C0C0} 
Factor                                                                & Weight  \\ \hline
Playing games over many long sessions                                 & 10     	\\ \hline
Being tolerant of frustration                                         & 9       \\ \hline
Desiring to modify or extend games in a creative way                  & 8       \\ \hline
Playing for the exhilaration of defeating (or completing) the game    & 7       \\ \hline
Being engaged in competition with oneself, the game and other players & 6       \\ \hline
Preferring games that have depth and complexity                       & 3       \\ \hline
Preferring violent/action games                                       & 1       \\ \hline
\end{tabular}
\caption{Table of relevant factors.}
\label{tab:relevantFactors}
\end{table}