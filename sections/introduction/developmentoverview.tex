\section{}
\subsection{Methodology}
\subsection{Identification of key topics}
In order to create a game which fulfil our problem statement\ref{introduction:problemstatement}, it is relevant to explore which gameplay concepts tie to which category of games. For example, both Dota2\cite{dota2} and Counter-Strike\cite{counterstrike} are considered highly competitive strategical games. However, whereas Counter-Strike has a relatively simple entry learning-curve and gameplay, Dota2 has an extremely high entry learning-curve. These differences in game design are relevant in terms of understanding what makes a game complex, strategical, fast / slow paced etcetera.
In other words, which gameplay-topics makes a game have advanced gameplay, and which topics supports player-developed strategical gameplay?

\subsection{What do other games do}\tododennis{I'm not even sure this should be here. It feels like we're touching relevant topics, but it is extremely difficult to explain it in short and precise sentences}
There are multiple ways of achieving complex gameplay. It can be achieved with both a relatively simple gameplay or very complex. In Counter-Strike two teams, Terrorists and Counter-Terrorists, consisting of 5 player each go head to head in a first-person shooter combat. The game is round based, and a team can win a round by either killing all other opponents. Further the terrorist team can win by exploding a bomb at designated targets, and the counter-terrorists can win by denying this. Further the game has an economy aspect, you get more money for winning a round, and less money for loosing a round. This money can be spent on weapons of varying strength. 
The first team to get to 16 rounds out of 30 (15 rounds played on both sides) win. Each round has a time limit of 1 minute 45 seconds, which makes the game rather fast paced. 
The strategical aspect of the game is mainly situated around complex player-developed tactics for various levels, and managing economy. 


Dota2 also consists of two teams of 5 players. They go head to head in a large map, where levelling, selecting skills, buying items to enhance character performance and team work is the main focus. Further there is a massive 110 different heroes, all with 4 different abilities. Each game takes anywhere from twenty minutes to over an hour. 
Strategical aspects involve managing gold-income from player kills, trying to get ahead in levels to become more powerful, and adapting to what the other team does. Further each game starts with a picking phase, where either team take turns to ban heroes and render them not pickable for the game.     

Both these games has extremely competitive communities with large tournaments and large prize-pools. In that way they are similar, but at the same time they are very different in the complexity of the game mechanics they use. Dota2 has many useable actions that do different things, where Counter-Strike has very few. Dota2 is slow, counter-strike is fast. Neither has elaborate stories, but still has a setting for the game world, and both are driven by cooperation by a team.

\subsection{Our key topics}
The game developed will have some main topics which it revolves around. They are listed and elaborated below.

\begin{itemize}
\item Competative gameplay vs Casual gameplay
\item Team work vs Individuality
\item Quick games vs Slow games
\item Story driven vs Gameplay driven
\item Fast learning-curve vs High learning-curve
\end{itemize}  

\subsection*{Focus topics for the developed game}
The game developed should definitely be oriented around a competitive gameplay. Competitive games tend to allow for more strategical playing and team work. Competitive games also tend to engage a player for a longer period of time because the re-playability is high, whereas story driven games often take the player on a journey after which the game has ended. The game should also favor team-oriented play, because this further heightens the strategical aspect of the game.
It also allows for more intricate strategies compared to a single-player game. Lastly, it also supports motivation of creating a synchronized multi-player game on mobile devices. 

The game should have a quick entry level, 	that is, it should be relatively easy to understand what you have to do, and how you do it. Actions should be kept to a minimum as we're on a mobile device, but the control scheme should still allow for more complex mechanics than most traditional mobile games with tap/tap-and-hold/swipe games like Angry Birds\cite{angrybirds} or Hayday\cite{hayday}. 

The game should have an overwhelming focus on gameplay rather than story, because it is the gameplay which fosters competitive play and replayability. 

Most popular mobile games have a short play-session period as described in the motivational chapter\ref{introduction:motivation}. As we would like to try and engage players for a longer time and play with team-mates in real time, the game will have a focus on longer play sessions.

This makes the main topics for our game:
\begin{itemize}
\item Competitive gameplay
\item Team work
\item Slower gameplay with a longer play-session
\item Easy entry-learning curve, but with advanced mechanics
\end{itemize}

\subsection{Prioritized issue list and milestones}