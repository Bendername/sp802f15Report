\section{Our game: GAME\_NAME}\todobenjamin{This name is not ideal.}

Whilst obviously designing a game which is fun, it is important to us that we try to understand the concepts of cognitive flow outlined in \ref{gamedesign:cognitiveflow} such that we have a psychological scientific foundation on which our game design lies. 
So with that in mind, a game idea will be developed and each element will be justified with a cognitive-flow-concept which it empowers. A list of all major concepts of the game can be found in the following section, \ref{gamedesign:maingameconcepts}. 
The game idea will then undergo rapid prototyping and testing, and adjustments will be made according to what is \emph{fun} and what is not.  

\subsection{Main game concepts}\label{gamedesign:maingameconcepts}
There are a few obstacles which has to be addressed in the game design. First of all, the game has to fulfill the initial goals outlined in the problem statement in \ref{sec:motivation}. These could be formalised as such:

\begin{itemize}
\item The game has to have advanced gameplay
\item The game has to have strategical aspects
\item The game has to be multiplayer
\end{itemize}

\subsubsection{Main goal of the game}
The main goal in the game, is to reach the highest highscore. This requires each level to be somewhat normalized, such that if group x beats the level, they had the same total reachable score as another group. That is, if level one has 10 zombies group x, then level one should contain an obtainable score equalling 10 zombies for group y. The entire score system is explained in \ref{gamedesign:ourgame:scoresystem}. The score is achieved by surviving a wave of enemies or getting as far as you can at a wave.\todoanders{Dennis would you please specific this further not sure I understand completely.} You lose either by the entire team dying, or the monsters killing a special crate of antidotes in the center of your base. Hence the goal is to protect your base, whilst getting as far as you can without dying.

\subsubsection{Advanced Gameplay}
According to Oxford Dictionary gameplay is defined as \textit{''the plot of a computer or video game or the way that it is played''}.
In games such as \textit{Candy Crush} \todoanders{Source here} the gameplay consist of destroying candy by swapping candy to align three pieces of specific candy, this is achieved by swiping the finger across the screen.
This is a very simple gameplay in comparison to games such as \textit{Starcraft} \todoanders{Source needed} which has a lot of controls and a complex plot.
More advanced controls schemes such as the ones you find in \textit{Starcraft} requires many buttons to give the user full control of the game.
This is a problem on mobile devices since the screen is already limited in size, and placing to many buttons will create to much clutter on the small screen.

To reduce the screen clutter a game genre with simpler control scheme will be necessary.
The game should however still have the complex plot to withhold the advanced gameplay.

Many games genres fits into this category therefore the top-down shooter is chosen.
This is chosen because the controls in all basics consist of moving and shooting in a two dimensional space.
Furthermore an RPG elements, such as experience, traits, and classes, should be added to help creating the complex plot.

\subsubsection{Strategical aspects}
The Oxford dictionaries definition of strategy is \textit{``A plan of action designed to achieve a long-term or overall aim''}.\cite{strategyOx}
Giving the player the ability to make meaningful choices would allow for the player to create strategies to achieve its goal.
As such, we can define the strategic aspect of a video game as a series of meaningful choices.\\

\textbf{The player}\\
The player controls a single character of a class of their choosing, see \ref{gamedesign:ourgame:classes}. 
This character can move freely in the game world.
When killing a zombie, the player will receive experience.
As experience is gained, the player will eventually level up, and allow to player to customize its character by choosing a set of traits, see \ref{gamedesign:ourgame:traits}.
The character is equipped with a set of weapons, which are obtained through crafting, see \ref{gamedesign:ourgame:crafting}.
Selecting traits and weapons crafting not only makes the player feel a sense of progression, but are meaningful choices in regards to the players strategy.\\

\textbf{Classes}\label{gamedesign:ourgame:classes}\\
The classes in the game are rather simple, yet have their own distinct role.
It is emphasized in the design that all classes are viable for beating the game.
The goal of the classes are to cater to different players play-style.
Some may feel most at home being close up and person with the zombies, while others may feel more engaged and entertained by standing back and having a slightly slower pace.
The classes in the game include:

\begin{itemize}
\item Jack 'Happy' Tricker - Appealing to tanky, ''in your face'' kind a player
\item 'Mending' Medy - Appealing to supportive types, who like to stand a bit back
\item Hunting Hank - Appealing to high stress players who enjoy low life with large damage and lots of movement
\item Mad Joe - Appealing to the average player. \todomichael{Who's the average player?}
\end{itemize}

Common for all of these are that none should feel boring. 
The game is a shooter, and everyone should shoot, in what way depends on the players choice of class.\\

\textbf{Traits}\label{gamedesign:ourgame:traits}\\
The trait system allows for customization of the chosen character, and thus play-style and strategy the player intends to use.
There is a shared pool of traits, including:
\begin{itemize}
\item Fast learner - Gain double experience points
\item Boom! Headshot!  - Gain 20\% critical hit chance
\item Just Can?t Get Enough - More zombies will spawn
\end{itemize} 
A full list of these can be found in the appendix \tododennis{Remember to add an exhaustive list}.
Additionally, a list of class specific traits can be chosen at given level intervals. 
These traits are only obtainable by the class, and each time a choice has to be made as to which traits the player wish to continue with.\\

\textbf{Weapons}\label{gamedesign:ourgame:weapons}\\
The weapons are another place of customization.
There are many different weapons, ranging from silly weapons like \emph{The Squirt Gun} to more common weapon types like the \emph{Assault Rifle}.
Each weapon has multiple upgrades, which can be bought in the shop.
Further a choice has to be made each round; \emph{Do I save up more resources for the next weapon, or do I think we can manage one more wave?} A complete list of weapons can be found in \appendix{}\tododennis{add weapon appendix}\\

\textbf{Buy system and Resources}\label{gamedesign:ourgame:crafting}\\
The resources is a requirement to buy new weapons.
If a player has the given resources required to \emph{craft} a weapon, he will receive it.
Any weapon can cost x amount of three different resources:
\begin{itemize}
\item Junk - The most common resource
\item Electrical Parts - Somewhat rare resource
\item Alien Tech - Very rare resource \tododennis{the name may have to be changed}
\end{itemize}
These resources are obtained by a mixture of shooting zombies and completing objectives.\\

\textbf{Objectives}\label{gamedesign:ourgame:objectives}\\
The objectives serves multiple purposes in the game. It allows a dynamic stress-factor and a selective skill based event. 
That is, the game is scalable to what the players skill level is, which is important in terms of the player being able to reach its goals \ref{gamedesign:cognitiveflow}. 
An objective is an event which forces the player to make a risk/reward evaluation. 
For instance, an objective may start on a wave, where a special monster is trying to flee with some precious cargo. 
The player can then attempt to kill this monster to reap the rewards, or stay in base and make sure it is not lost to zombies.
Further, each object should favor a certain subset of the classes, which enhances the value of the players choice of class.\\

\textbf{Multiplayer}\label{gamedesign:ourgame:multiplayer}\\
Making the game a team based game will allow for even further strategic depth, as the player will now have to coordinate defenses and resources with its friends.
It creates more depth in the class system, as the players choice of team composition will impact the strategy he or she employs, as well as the weapons of choice.
Taking traits that will benefit the entire team and not just the player itself might be advantageous in certain scenarios, whilst focusing on the players individual power is advantageous in others.

\subsubsection{Multiplayer}
\input{sections/gamedesign/multiplayer}

As the main focus of the game is multiplayer, an aspect which has to be addressed is the difficulty. What one player finds difficult may feel easy for another player. This can be solved somewhat by making it team oriented, and making different roles for different types of people. The tricky part is making sure no team composition is enforced, making people play roles they do not feel they are enjoying. 
Secondly, having a set of player roles allows for various game strategies, depending on which roles has been selected. As such meta strategies within the game may appear. These are strategies which maximize the possibility for \emph{winning}, whether it is reaching the furthest level or getting the highest highscore. 

\subsection*{The player}
 An individual player consists of a single character. This character can move freely in the game world without being constrained by what other players are doing. When killing a zombie, the player will receive experience. When a set amount of experience is reached, the player will gain a level. Gaining levels will allow the player to customize his character by choosing a set of traits, see the traits section \ref{gamedesign:ourgame:traits}. The player wears a set of weapons, which are obtained through crafting, see \ref{gamedesign:ourgame:crafting}. Both gaining levels, trait selection and crafting allow for the player to feel he is progressing on multiple levels in the game. When progressing, the player receives some form of feedback, basically it should feel \emph{awesome} to level up.

\subsection*{Traits}\label{gamedesign:ourgame:traits}
The traits of a character allow for deep customization and advanced strategies in the game. There is a shared pool of traits, including:
\begin{itemize}
\item Fast learner - Gain double experience points
\item Boom! Headshot!  - Gain 20\% critical hit chance
\item Just Can’t Get Enough - More zombies will spawn
\end{itemize} 
A full list of these can be found in the appendix \tododennis{Remember to add an exhaustive list}. Additionally, a list of class specific traits can be chosen at given level intervals. These traits are only obtainable by the class, and each time a choice has to be made as to which traits the player wish to continue with. This again opens up for depth in game play allowing for strategies depending on amount of a given class and amount of players in a game. The class traits can be seen under the class review for each class \ref{gamedesign:ourgame:classes}.

\subsection*{Buy system and Resources}\label{gamedesign:ourgame:crafting}
The resources is a requirement to buy new weapons. Any weapon can cost x amount of three different resources:
\begin{itemize}
\item Junk - The most common resource
\item Electrical Parts - Somewhat rare resource
\item Alien Tech - Very rare resource \tododennis{the name may have to be changed}
\end{itemize}
These resources are obtained by a mixture of shooting zombies and completing objectives. Objectives are described in section \ref{gamedesign:ourgame:objectives}. If a player has the given resources required to \emph{craft} a weapon, he will receive it. Weapons are described in more detail in the \ref{gamedesign:ourgame:weapons} section.

\subsection*{Classes}\label{gamedesign:ourgame:classes}
The classes of \emph{Last Stand} are rather simple, yet have their own distinct role. It is emphasized in the design that the classes are not necessary in order to beat the game, however, there may be a class composition which makes some tasks easier, or makes reaching the top of the leaderboards easier. 
Further the goal of the classes are to cater to different players playing-style. Some may feel most at home being up in the zombies faces, shooting everywhere, while others may feel more engaged and entertained by standing back and having a slightly slower pace. This difference allows for different types of players with different types of stress-levels to play together, in a uniformly scaled game difficulty such that no player is left feeling overly stressed, and no player is left feeling bored. The classes in the game include:

\begin{itemize}
\item Jack 'Happy' Tricker - Appealing to tanky, ''in your face'' kind a player
\item 'Mending' Medy - Appealing to supportive types, who like to stand a bit back
\item Hunting Hank - Appealing to high stress players who enjoy low life with large damage and lots of movement
\item Mad Joe - Appealing to the average player.
\end{itemize}

Common for all of these are that none should feel boring. The game is a shooter, and everyone should shoot. They may not all do the same damage, but they are all capable of doing damage, and the healer is e.g. not forced to stand around and heal at all times. Instead she will heal the group (as the only one), when shooting monsters. More detailed class definitions can be found in appendix \ref{}\tododennis{Make this appendix with classes}. 

\subsection*{Weapons}\label{gamedesign:ourgame:weapons}
The weapons are another place of great customization. There are many different weapons, ranging from silly weapons like \emph{The Squirt Gun} to more common weapon types like the \emph{Assault Rifle}. Each weapon has multiple upgrades, which can be bought in the shop. The multitude of games also support a wide array of strategies, and depending on the team composition some unlockables may be reachable or not. Further a choice has to be made each round; \emph{Do I save up more resources for the next weapon, or do I think we can manage one more wave?} A complete list of weapons can be found in \appendix{}\tododennis{add weapon appendix} 

\subsection*{Objectives}\label{gamedesign:ourgame:objectives}
The objectives serves multiple purposes in the game. It allows a dynamic stress-factor and a selective skill based event. That is, the game is scalable to what the players skill level is, which is great in terms of the player being able to reach his goals \ref{gamedesign:cognitiveflow}. An objective is an event which forces the player to make a risk/reward evaluation. For instance, an objective may start on a wave, where a special monster is trying to flee with some precious cargo. The team can opt in to attempt to kill this monster, reaping the rewards, or stay in base and make sure they are not loosing the main base to zombies, and dying because the antidote crate is killed. Further the objectives should attempt to make specific roles more favourable compared to others. For instance the monster may be more vulnerable to explosives, making \emph{Mad Joe} the obvious choice. Or the monster may have an aura of poison, making it inherit that at least two players go to fight him, preferably one of them being \emph{Mendy} the medic. 

