The Oxford dictionaries definition of strategy is \textit{``A plan of action designed to achieve a long-term or overall aim''}.\cite{strategyOx}
Giving the player the ability to make meaningful choices would allow for the player to create strategies to achieve its goal.
As such, we can define the strategic aspect of a video game as a series of meaningful choices.\\

\textbf{The player}\\
The player controls a single character of a class of their choosing, see \ref{gamedesign:ourgame:classes}. 
This character can move freely in the game world.
When killing a zombie, the player will receive experience.
As experience is gained, the player will eventually level up, and allow to player to customize its character by choosing a set of traits, see \ref{gamedesign:ourgame:traits}.
The character is equipped with a set of weapons, which are obtained through crafting, see \ref{gamedesign:ourgame:crafting}.
Selecting traits and weapons crafting not only makes the player feel a sense of progression, but are meaningful choices in regards to the players strategy.\\

\textbf{Classes}\label{gamedesign:ourgame:classes}\\
The classes in the game are rather simple, yet have their own distinct role.
It is emphasized in the design that all classes are viable for beating the game.
The goal of the classes are to cater to different players play-style.
Some may feel most at home being close up and person with the zombies, while others may feel more engaged and entertained by standing back and having a slightly slower pace.
The classes in the game include:

\begin{itemize}
\item Jack 'Happy' Tricker - Appealing to tanky, ''in your face'' kind a player
\item 'Mending' Medy - Appealing to supportive types, who like to stand a bit back
\item Hunting Hank - Appealing to high stress players who enjoy low life with large damage and lots of movement
\item Mad Joe - Appealing to the average player. \todomichael{Who's the average player?}
\end{itemize}

Common for all of these are that none should feel boring. 
The game is a shooter, and everyone should shoot, in what way depends on the players choice of class.\\

\textbf{Traits}\label{gamedesign:ourgame:traits}\\
The trait system allows for customization of the chosen character, and thus play-style and strategy the player intends to use.
There is a shared pool of traits, including:
\begin{itemize}
\item Fast learner - Gain double experience points
\item Boom! Headshot!  - Gain 20\% critical hit chance
\item Just Can?t Get Enough - More zombies will spawn
\end{itemize} 
A full list of these can be found in the appendix \tododennis{Remember to add an exhaustive list}.
Additionally, a list of class specific traits can be chosen at given level intervals. 
These traits are only obtainable by the class, and each time a choice has to be made as to which traits the player wish to continue with.\\

\textbf{Weapons}\label{gamedesign:ourgame:weapons}\\
The weapons are another place of customization.
There are many different weapons, ranging from silly weapons like \emph{The Squirt Gun} to more common weapon types like the \emph{Assault Rifle}.
Each weapon has multiple upgrades, which can be bought in the shop.
Further a choice has to be made each round; \emph{Do I save up more resources for the next weapon, or do I think we can manage one more wave?} A complete list of weapons can be found in \appendix{}\tododennis{add weapon appendix}\\

\textbf{Buy system and Resources}\label{gamedesign:ourgame:crafting}\\
The resources is a requirement to buy new weapons.
If a player has the given resources required to \emph{craft} a weapon, he will receive it.
Any weapon can cost x amount of three different resources:
\begin{itemize}
\item Junk - The most common resource
\item Electrical Parts - Somewhat rare resource
\item Alien Tech - Very rare resource \tododennis{the name may have to be changed}
\end{itemize}
These resources are obtained by a mixture of shooting zombies and completing objectives.\\

\textbf{Objectives}\label{gamedesign:ourgame:objectives}\\
The objectives serves multiple purposes in the game. It allows a dynamic stress-factor and a selective skill based event. 
That is, the game is scalable to what the players skill level is, which is important in terms of the player being able to reach its goals \ref{gamedesign:cognitiveflow}. 
An objective is an event which forces the player to make a risk/reward evaluation. 
For instance, an objective may start on a wave, where a special monster is trying to flee with some precious cargo. 
The player can then attempt to kill this monster to reap the rewards, or stay in base and make sure it is not lost to zombies.
Further, each object should favor a certain subset of the classes, which enhances the value of the players choice of class.\\

\textbf{Multiplayer}\label{gamedesign:ourgame:multiplayer}\\
Making the game a team based game will allow for even further strategic depth, as the player will now have to coordinate defenses and resources with its friends.
It creates more depth in the class system, as the players choice of team composition will impact the strategy he or she employs, as well as the weapons of choice.
Taking traits that will benefit the entire team and not just the player itself might be advantageous in certain scenarios, whilst focusing on the players individual power is advantageous in others.