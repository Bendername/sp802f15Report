The design of our game should satisfy the conditions specified in Table \ref{tab:relevantFactors} as well as be a mobile game.
The game design must match players with the following charateristics:
\begin{enumerate}\label{gamedesign:selectionofgametype:importantstuff}
\item \textbf{Playing games over many long sessions.} The game should have long, repeatable game cycles.
\item \textbf{Being tolerant of frustration.} The game should be difficult in order to induce frustration.
\item \textbf{Desiring to modify or extend games in a creative way.} The game should be expandable by the users.
\item \textbf{Playing for the exhilaration of defeating (or completing) the game.} The game should reward the player when completed.
\item \textbf{Being engaged in competition with oneself, the game and other players.} The game should benchmark how well the player is doing and show that to the player.
\item \textbf{Preferring games that have depth and complexity.} The game should have advanced gameplay and complex strategical aspects.
\item \textbf{Preferring violent/action games.} The game should be a violent action game.
\end{enumerate}
By following these conditions, it is likely that our game design will cater to hardcore players, and thus satisfy our problem statement. 
Therefore these seven conditions are requirements for our game design.

These points will be further discussed in sections \ref{selectionofgametype:advancedgameplay}, \ref{selectionofgametype:frustration}, and \ref{sec:selectionofgametype:customcontent}.
Some are closely linked and will be coupled when discussed.

\subsection{Advanced Gameplay \& Complex Strategical Aspects}\label{selectionofgametype:advancedgameplay}
The topic of gameplay and strategy are very closely tied, since they both focus on making the player take choices which satisfies the player.
Advanced gameplay also allows for longer and more complex game cycles.

Games such as \textit{Candy Crush}\cite{candycrush} have strategic aspects tied to the fact that each move the player makes will cause a series of actions in the game.
The series of actions is somewhat calculable by the player, except for randomized events such as the new tiles which will fall down from above in \textit{Candy Crush}.
Given this calculation the player can make a move which he believes will take him closer to winning.
In essence this is not a very deep form of strategy because the player is very limited to the actions he can make to achieve the goal.

A game like \textit{Dota 2}\cite{Dota2} also has many random factors and contains multiple features which all adds to the strategical complexity.
The game consists of two teams of five players that go head to head on a large map where team work is the main focus.
Dota 2 consists of 110 different heroes, all with at least 4 different abilities.
Each game takes anywhere from twenty minutes to over an hour.
Strategical aspects involve managing gold-income from player kills, trying to get ahead in levels to become more powerful, and adapting to what the other team does.
Each game starts with a ``picking phase'' and the teams take turns to ban heroes from the game as well as pick the heroes they are going to play with.
This gives the players a lot of choices which can advance them closer to their goal of winning, and is a good example of a game with a complex strategical aspect.

\subsection{Frustration and competition}\label{selectionofgametype:frustration}
The game should be difficult in order to induce frustration, as mentioned in Table \tododaniel{ik rigtigt en table, ved sku ik lige hvad man skal kalde det tho} \ref{gamedesign:selectionofgametype:importantstuff}.
Therefore, the level of difficulty should be high, but still fit the players skill level in order to stay inside the flow zone as shown in Figure \ref{gamedesign:flowzone}.
The player should have a benchmark, such as a high score, of how well he is doing.
Properly challenging the player and presenting how well he is doing, means that the player will feel a sense of achievement for overcoming the challenge.
This can be further reinforced by ingame feedback and rewards.
The benchmarking can engage the player in competition with himself and the game, and other players.
Especially if the game is multiplayer, and automatically compares benchmarks between players.
Real-time multiplayer also enables interacting with the other players, which also further reinforces the competition.

\subsection{Custom content}\label{sec:selectionofgametype:customcontent}
The game should feature some way for players to create their own custom content, and possibly share it with other players. Areas in which this could be achieved is the ability to create custom levels and possibly scenarios in which you can play with your friends. The content should be distributed in an easy way, such that it requires little action from the participating players.
This will cater to the desire for extending or modifying the game.

\subsection{Mobile devices}\label{sec:selectionofgametype:mobiledevices}\tododaniel{Section snakker om mobile devices og hvordan de skal styres. example er så CS som er et pc game??}
To satisfy the limitation which comes with the mobile input, a game with a less complicated control scheme is required.
An example of a game with a simple control scheme could be a shooter, such as \textit{Counter-Strike}.\cite{counterstrike}
In \textit{Counter-Strike} two teams, Terrorists and Counter-Terrorists,
consisting of 5 players each go head to head in a first-person shooter combat.
The strategical aspect of the game is mainly situated around complex
player-developed tactics for various levels, weapon composition across the
team, and managing economy. This is possible without the use of a complex
control scheme, which for Counter-Strike is basically run, point, and shoot.
This indicates that a game in the shooter genre would
be a good candidate regarding the strategical aspect with interesting choices
and have the capability of have some depth and complexity.

Another candidate could be a top-down shooter because it only requires the
player to control movement and aim in a two dimensional space. Shooters in
general often satisfy the violence/action aspect, which is the case for
Counter-Strike.

