The type of game chosen should satisfy the conditions specified in Table \ref{tab:relevantFactors}.
Satisfying this conditions can be done as follows.
\begin{enumerate}
\item \textbf{Playing games over many long sessions.} The game needs to have long, repeatable game cycles.
\item \textbf{Being tolerant of frustration.} The game needs to be difficult in order to induce frustration.
\item \textbf{Desiring to modify or extend games in a creative way.} The game needs to be expandable by the users.
\item \textbf{Playing for the exhilaration of defeating (or completing) the game.} The game needs to reward the player when completed.
\item \textbf{Being engaged in competition with oneself, the game and other players.} The game needs to benchmark how well the player is doing and show that to the player.
\item \textbf{Preferring games that have depth and complexity.} The game needs to have advanced gameplay and complex strategical aspects.
\item \textbf{Preferring violent/action games.} The game needs to be violent.
\end{enumerate}


%In summary these conditions consist of:
%\begin{itemize}
%\item Synchronous multiplayer - Multiple players in the same game, who in real-time carry out actions which effect the game
%\item Mobile devices - The game should run with limited input interface and limited hardware compared to a PC
%\item Advanced gameplay - A series of interesting choices
%\item Complex strategical aspects - A series of meaningful choices which will take the players closer to the goal
%\end{itemize}

%\subsection{Multiplayer Types}
%There exist various games that use synchronous multiplayer and may handle the topic differently. 
%Some examples could be massive multiplayer online games in which there are a lot of players gathered on one server.
%These players do not necessarily interact with each other but have the possibility to do so, a classic example of this could be \textit{World of Warcraft}\cite{wow}. 
%Turn based games is another genre which can contain synchronous multiplayer, games such as \textit{Civilization} and \textit{Hearthstone}\cite{hearthstone} are examples of this. 
%In general multiplayer is applicable to must games, and it adds another level of complexity or redefine it as a whole. %wtf does this even mean

%\subsection{Advanced Gameplay \& Complex Strategical Aspects}
%The topic of gameplay and strategy are very closely tied, since they both focus on making the player take choices which satisfies the player.

%Games such as \textit{Candy Crush}\cite{candycrush} have strategic aspects tied to the fact that each move the player makes will cause a series of actions in the game.
%This series of actions is somewhat calculable by the player, except for randomized events such as the new tiles which will fall down from above in \textit{Candy Crush}.
%Given this calculation the player can make a move which he/she believes will take him closer to winning.
%In essence this is not a very deep form of strategy because the player is very limited to the actions he/she can make to achieve the goal.\\
%A game like \textit{Dota 2}\cite{Dota2} also have many random factors but also contains multiple features which all adds to the strategical complexity.
%The game consists of two teams of five players each. 
%They go head to head in a large map, where levelling, selecting skills and buying items, all to enhance character performance, and team work is the main focus. 
%Further there is a massive 110 different heroes, all with at least 4 different abilities. 
%Each game takes anywhere from twenty minutes to over an hour. 
%Strategical aspects involve managing gold-income from player kills, trying to get ahead in levels to become more powerful, and adapting to what the other team does. 
%Further each game starts with a picking phase, where the teams take turns to ban heroes from the game as well as pick the heroes they are going to play with.
%This gives the players a lot of choices which can advance them closer to their goal of winning, and is a good example of a game with a complex strategical aspect.

%\subsection{Mobile devices}
%To satisfy the limitation which comes with the mobile input, a game with a less complicated control scheme is required.
%A candidate could be a shooter, an example of which could be \textit{Counter-Strike}. \cite{counterstrike}
%In \textit{Counter-Strike}two teams, Terrorists and Counter-Terrorists, consisting of 5 players each go head to head in a first-person shooter combat. 
%The game is round based, and a team can win a round by killing all other opponents. 
%Further the terrorist team can win by exploding a bomb at designated targets, and the counter-terrorists can win by denying this. 
%Further the game has an economy aspect, you get money for various events such as killing an opponent or winning a round.
%Losing a round also grants money, but less than winning.
%This money can be spent on weapons of varying strength.
%The first team to get to 16 rounds out of 30 (15 rounds played on both sides) win. 
%The strategical aspect of the game is mainly situated around complex player-developed tactics for various levels, weapon composition across the team, and managing economy, without complex control scheme.\\
%This indicates that a game in the shooter genre would be a good candidate regarding the strategical aspect.