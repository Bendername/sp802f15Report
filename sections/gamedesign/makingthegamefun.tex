An important part of creating a game is to make it attractive to play, and one way of doing so is to make it fun.
According to Sid Meier\cite{GDC2012}, other contributions to attractive gameplay could be a series of interesting decisions.
Sid Meier describes four characteristics of interesting decisions\cite{GDC2012}:
\begin{itemize}
	\item Trade-offs - Do I buy this big sword, but spend a lot of gold doing so?
	\item Situational - For this track, do I go with the car with good handling or the car with most horsepower?
	\item Personal - Do I build a lot of defensive units and play defensively, or do I play aggressive and try to rush down my enemy?
	\item Persistence - My decisions should be informed and have an impact on the game
\end{itemize}

In order to have the basics of a fun game design, the game loop should enable the player to make interesting decisions that affect the game.
Another thing which makes a game attractive to play is immersion which happens when players experience a loss of self-awareness and lose track of time.
This can be described with cognitive flow.

\subsection{Cognitive Flow}\label{makingthegamefun:cognitiveflow}
\emph{This section is largely based on an article by Sean Baron, User Experience Researcher at Microsoft Game
Studios\cite{baron}.}

Cognitive Flow is a psychological term for being immersed in a game, or ``in the zone''.

To increase the chance of reaching this cognitive flow, four main states in terms of game to player interaction has to
be promoted.  
These four states are 
\begin{itemize}
    \item \emph{Concrete goals with manageable rules}
    \item \emph{Goals that fit player capabilities}
    \item \emph{Clear and timely feedback}
    \item \emph{Removal of distractions from player focus}
\end{itemize}

In order to maximize the fun of the game, these four states must be considered when designing the core game loop, and thus the decisions available to the players.

\subsubsection{Concrete goals with manageable rules}
The player must always know how to accomplish the next goal of their choice, whether it is to get the an upgraded weapon, or simply get from A to B.
When players are unaware of what to do next, they are more likely to stop playing.
Following this, the player should be presented with relevant information without distracting him from the game.

\subsubsection{Goals that fit player capabilities}
A game should not be too easy, nor too hard.
If the player is at any end of the spectrum, he will either feel bored or too stressed.
The sweet spot is at just the right balance between the two, as this in fact heightens the performance of the player, see Figure \ref{gamedesign:flowzone}.
\begin{figure}
    \includegraphics{figures/gamedesign/flowZone}
    \caption{Illustration of the flow zone.}
    \label{gamedesign:flowzone}
\end{figure}

It is important to note that the skill level of a player is highly subjective.
Any new game mechanic or game changing feature also has to be taught to the player, if it deviates from the conventional scheme for the genre.
That is, if the game is a first person shooter, then you do not have to teach
the player to aim if it follows the normal conventions. However, if you can suddenly
jump on walls or walk upside down, then this has to be taught in a setting
where teaching is the only focus.

\subsubsection{Clear and timely feedback}
If a player is unsure about whether he has done something correct, then the feedback timing is off.
If the player shoots a zombie and should receive experience points for doing so, then it should happen in a timely manner after the zombie has been killed, such that the player can link the reward with the challenge overcome.
Furthermore, there has to be short-term mechanisms which convey this information, as well as long-term mechanisms which convey goals spanning longer durations,
such as collecting a key to complete a level, or gathering multiple items to craft a special item.

\subsubsection{Remove distractions from player focus}\label{introductions:makingthegamefun:removedistractionsfromplayerfocus}
The main focus for the player is for him to be engaged in the game.
If the player has to stop up and pay attention to information, it obstructs gameplay.
Of course there are exceptions to this rule, like giving puzzles which require the player to think in order to solve it.
The trick is making this information precise and concise, such that the
information flow from the game to the player is not cluttered with unnecessary or hard to understand information.

This concludes Sid Meier's definitions for creating a fun game and the
cognitive flow theory.
We take these theories into consideration when designing our game.
